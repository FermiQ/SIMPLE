\documentclass{article}
\usepackage{amsmath}
\usepackage{hyperref}
\usepackage{graphicx}
\begin{document}
\title{SIMPLE's implementation of Butterworth function/kernel used in non-uniform refinement}
\maketitle

\section{Continuous Butterworth-based optimization}
\noindent Let $x$ be a frequency value and $\theta$ the cut-off frequency, then the Butterworth polynomial of order $n$-th is estimated \cite{ButterworthWiki} as
\begin{equation}\label{eq1}
B_n(x,\theta) = \sum_{k = 0}^n a_n s^k, \quad s := \frac{ix}{\theta}, \quad i = \sqrt{-1}.
\end{equation}
The derivative of $B_n$ w.r.t $\theta$ is:
\begin{align}
\frac{dB_n}{d\theta}(x,\theta) &= \frac{dB_n}{ds}\frac{ds}{d\theta} \\
&= \left( \sum_{k = 1}^n ka_ns^{k-1} \right) \left(-\frac{ix}{\theta^2}\right) \\
&= -\frac{1}{\theta}\sum_{k=1}^n ka_ns^k
\end{align}
The Butterworth transfer function $H_n(x,\theta)$ is
\begin{equation}\label{eq5}
H_n(x, \theta) = \left| K_n(x,\theta) \right|, \quad K_n(x, \theta) = \frac{1}{B_n(x, \theta)}
\end{equation}
The derivative of $H_n$ w.r.t. $\theta$ is
\begin{align}\label{eq6}
\frac{dH_n}{d\theta} = \frac{d}{d\theta}\left| K_n(x,\theta) \right| = \frac{\text{Re}\left(K_n(x,\theta)\overline{\frac{dK_n}{d\theta}}\right)}{\left| K_n(x,\theta) \right|}
\end{align}
where
\begin{equation}\label{eq7}
\frac{dK_n}{d\theta} = -\frac{1}{B_n^2}\frac{dB_n}{d\theta}
\end{equation}
The convolution of $H_n$ with a model function $m(x)$ is
\begin{equation}\label{eq8}
f(x,\theta) = H_n(x,\theta) \star m(x) = \int H_n(\zeta, \theta)m(x-\zeta)d\zeta
\end{equation} 
The derivative of $f(x,\theta)$ w.r.t $\theta$ is
\begin{equation}\label{eq9}
\frac{df}{d\theta} = \int\frac{dH_n}{d\theta}(\zeta,\theta)m(x-\zeta)d\zeta = \frac{dH_n}{d\theta}(x,\theta) \star m(x)
\end{equation}
The cost function is defined as:
\begin{equation}\label{eq10}
F(\theta) = ||f(x,\theta) - f_0||^2 = \int (f(x,\theta) - f_0(x))^2dx
\end{equation}
with the derivative
\begin{equation}\label{eq11}
\frac{dF}{d\theta} = \int \frac{df}{d\theta}(x,\theta) \cdot (f(x,\theta) - f_0(x))dx
\end{equation}
In SIMPLE's refinement process, $f_0$ is the odd and $m(x)$ is the even.\\

\noindent If one requires voxel-wise cut-off frequency $\theta_x$, then the cost function can be defined as
\begin{equation}\label{eq12}
F(\theta_x) = \sum_x (f(x,\theta_x) - f_0(x))^2dx
\end{equation}
with the derivative
\begin{equation}\label{eq13}
\frac{dF}{d\theta} = \int \frac{df}{d\theta}(x,\theta) \cdot (f(x,\theta) - f_0(x))dx
\end{equation}

\newpage
\section{Discrete Butterworth-based search}
For each voxel $x$, we would like to search for the cut-off frequency $\theta(x)$ to minimize the following cost function
\begin{equation}\label{eq14}
\theta_{\min}(x) = \min_{\theta}\sum_\xi W_\rho(\xi-x) \, |(B_{\theta(x)} \star \text{odd})(\xi) - \text{even}(\xi) |
\end{equation}
where
\begin{itemize}
\item $B_{\theta(x)}$: Butterworth function, depending on the current voxel $x$, with cut-off frequency $\theta$
\item $W_\rho$: a window function, which is positive and integrates to 1, with spatial extent $\rho$.\\
Currently in the codes, this window function is a linear function with is 1 at $x$ and linearly decreases to zero at $x \pm \rho$. It is normalized to have an energy of 1.
\end{itemize}
For a Butterworth function with cut-off frequency $\theta$, its FT has some support $\text{sup}_\theta$. In the figure below, the cut-off frequency is $\theta = 10$ and the $\text{sup}_\theta = 13$.
\includegraphics[width=\textwidth]{pics/Butterworth_CutoffFreq_FTSup}
The correspondence between $\text{sup}_\theta$ and its cut-off frequency can be estimated by the following line in the codes: theta = a*exp(b*sup) + c*exp(d*sup).\\

\noindent Implementation of equation \eqref{eq14} has the following steps:
\begin{itemize}
\item[1.] Divide the interval $[\text{sup}_{\min}, \text{sup}_{\max}]$ into $N_\text{sup}$ equi-distant intervals. For example, in the figure below, $\text{sup}_{\min} = 4$, $\text{sup}_{\max} = 36$ and $N_\text{sup} = 4$.
\includegraphics[width=\textwidth]{pics/EquiSup}
For each $\text{sup}$, compute the corresponding cut-off frequency and compute the corresponding Butterworth kernel.
\item[2.] At each pixel, search over all the theta's in step 1 to see which one minimize equation \eqref{eq14}.
\item[3.] After finding all the optimized theta's at each pixel, denoising the odd/even with the optimized Butterworth kernel.
\end{itemize}

\begin{thebibliography}{9}
\bibitem[1]{ButterworthWiki} \url{https://en.wikipedia.org/wiki/Butterworth_filter}
\end{thebibliography}
\end{document}